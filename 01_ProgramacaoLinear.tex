\documentclass[t]{beamer}

% Load general definitions
\input{preamble.tex}

% Specific definitions
\title[]{Pesquisa Operacional}
\subtitle[]{Programação Linear}
\author[]{Patrícia Lucas\\{\footnotesize }}
\institute{Bacharelado em Sistemas de Informação \\ IFNMG  - Campus Salinas}
\date{\scriptsize Salinas\\Dezembro 2021}

\begin{document}

% cover page
\setbeamertemplate{footline}{}
\begin{frame}

\begin{center}
\includegraphics[width=.15\textwidth]{}
\end{center}
  \titlepage
  \begin{tikzpicture}[remember picture,overlay]
  \node[anchor=south east,xshift=-5pt,yshift=5pt] at (current page.south east) {\tiny Versão 1.2021};
  \node[anchor=south west,yshift=0pt] at (current page.south west) {\includegraphics[width=.25\textwidth]{Logos/salinas_horizontal_jpg.jpg}};
  \end{tikzpicture}  
\end{frame}

% Main slides
%==================================


\begin{ftst}{Visão geral}{Programação Linear}
Um problema de PL é descrito por três componentes importantes:
\small
\begin{itemize}
    \item Variáveis de decisão, que representam efetivamente a decisão que deve ser tomada no problema modelado.
    \vone
    \item Função objetivo, que representa em um valor numérico o benefício ou custo associado às decisões que devem ser tomadas. E a função que deve ser maximizada ou minimizada.
    \item Restrições, que representam a limitação dos recursos do mundo real. As restrições impõem que a solução deve obedecer certas regras.
\end{itemize}
\vone
\normalsize
Em um problema de PL, a função objetivo e as restrições são sempre lineares e as variáveis
de decisão são sempre variáveis reais (possivelmente dentro de um intervalo).

\end{ftst}

%==================================

\begin{ftst}{Formalização}{Programação Linear}

Uma programação linear (PL) é definida como um um problema de maximizar ou minimizar uma função linear (ou afim) sujeita a um número finito de restrições lineares. Considerando o exemplo:
    \begin{equation*}
        \begin{align*}
          \text{max}\ \ 3x_1 + 2x_2 - x_3 + 5 & \\ 
          \text{sujeito a} \ \ x_1 + x_2 \leqslant 9 & \\ 
           x_3 \leq 3 & \\ 
           x_1, x_2, x_3 \geq 0 & 
        \end{align*}
    \end{equation*}
\vone
\small
Estamos maximizando a função objetivo $f(x_1,x_2,x_3) = 3x_1 + 2x_2 - x_3 + 5$ sujeito às restrições $x_1 + x_2 \leqslant 9$, $x_3 \leq 3$, $x_1 \geq 0$, $x_2 \geq 0$, $x_3 \geq 0$.

\end{ftst}

%==================================

\begin{ftst}{Formalização}{Programação Linear}

\textbf{Importante:} uma restrição linear é sempre uma inequação da forma:

\begin{equation*}
    f(x) \leq \beta, \ \ f(x) \geq \beta, \ \ f(x) = \beta
\end{equation*}
onde $x$ é um vetor de variáveis e $\beta$ é um escalar. Note que,
\begin{equation*}
    3x_1 + 5x_2 - x_3 + x_4 < 5
\end{equation*}
não é uma restrição linear, já que a desigualdade é estrita.
\vone
\small
- Uma \textbf{solução} é uma atribuição de valores às variáveis ($x1$, $x2$, $x3$).
\vone
- Uma solução é \textbf{viável} se possui a propriedade de que todas as restrições são satisfeitas. 
\vone
- Uma solução é \textbf{ótima} se é viável e maximiza (ou minimiza se o problema for de minimização) a função objetivo.

\end{ftst}

%==================================

\begin{ftst}{Modelagem: exemplo 1}{Programação Linear}
\footnotesize
Suponha uma empresa que produza 4 tipos de produto. A empresa possui duas máquinas diferentes e a produção de cada produto requer horas em ambas as máquinas, além de horas operacionais e horas em um processo de controle de qualidade. A tabela abaixo especifica, para cada produto, quantas horas são necessárias em cada máquina/atividade. A tabela inclui também o preço de venda (assuma demanda infinita):

\begin{table}[]
\centering
\footnotesize
\begin{tabular}{|c|c|c|c|c|c|}
\hline
Produto & Máquina 1 & Máquina 2 & Operacional & Qualidade & Preço de venda \\ \hline
1       & 11        & 4         & 8           & 7         & 300            \\ \hline
2       & 7         & 6         & 5           & 8         & 260            \\ \hline
3       & 6         & 5         & 5           & 7         & 220            \\ \hline
4       & 5         & 4         & 6           & 4         & 180            \\ \hline
\end{tabular}
\end{table}

Por mês, a máquina 1 pode funcionar no máximo 700 horas, e a 2 por no máximo 500 horas. A empresa pode comprar no máximo 600 horas de trabalho operacional ao custo de 8 reais a hora, e 650 horas de controle de qualidade ao custo de 6 reais a hora. Quantos itens de cada produto a empresa deve produzir de forma a maximizar seu lucro?
\end{ftst}

%==================================

\begin{ftst}{Modelagem: exemplo 2}{Programação Linear}
\footnotesize
Considere o seguinte problema de demanda, armazenamento e distribuição. Uma companhia local, Salinas Insumos, de revenda de insumos agrícolas prevê que nos próximos meses, a demanda por seu principal insumo seja a seguinte:

\begin{table}[]
\centering
\footnotesize
\begin{tabular}{|c|c|c|c|c|}
\hline
Mês               & 1    & 2    & 3    & 4    \\ \hline
Demanda em litros & 5000 & 8000 & 9000 & 6000 \\ \hline
\end{tabular}
\end{table}

No começo de cada mês, esta empresa pode comprar este insumo de um distribuidor regional pelos seguintes valores:

\begin{table}[]
\centering
\footnotesize
\begin{tabular}{|c|c|c|c|c|}
\hline
Mês              & 1    & 2    & 3    & 4    \\ \hline
Custo por litros & 0.75 & 0.72 & 0.92 & 0.90 \\ \hline
\end{tabular}
\end{table}

A Salinas Insumos possui um tanque de armazenamento de 4000 litros, que atualmente já contém 2000 litros. A empresa deseja saber quantos litros de insumo deve comprar no começo de cada mês para suprir a demanda e ao mesmo tempo minimizar seus custos. Note que se o insumo é comprado e revendido imediatamente, não é preciso armazená-lo no tanque. Somente o excedente para o mês seguinte é armazenado. Para simplificar, assumimos que o custo de armazenamento é zero. 
\end{ftst}


%==================================

\begin{ftst}{Modelagem: exemplo 3}{Programação Linear}
Considere a seguinte tabela nutricional de alguns tipos de comida:

\begin{table}[]
\footnotesize
\centering
\begin{tabular}{|c|c|c|c|c|c|}
\hline
Comida       & Preço & Calorias & Gordura & Proteína & Carboidratos \\ \hline
Cenoura      & 0.14  & 23       & 0.1     & 0.6      & 6            \\ \hline
Batata       & 0.12  & 171      & 0.2     & 3.7      & 30           \\ \hline
Pão Integral & 0.20  & 65       & 0.0     & 2.2      & 13           \\ \hline
Queijo       & 0.75  & 112      & 9.3     & 7.0      & 0            \\ \hline
Amendoim     & 0.15  & 188      & 16.0    & 7.7      & 2            \\ \hline
\end{tabular}
\end{table}
\small
Uma nutricionista deseja montar um cardápio que minimize os custos diários, ao mesmo tempo que as seguintes demandas nutricionais são satisfeitas:
\begin{itemize}
    \item pelo menos 2000 calorias
    \item pelo menos 50g de gordura
    \item pelo menos 100g de proteína
    \item pelo menos 250g de carboidratos
\end{itemize}
\end{ftst}

%==================================

\begin{ftst}{Forma padrão de igualdades}{Programação Linear}

Até agora, trabalhamos com a forma canônica de formulação de um problema de PL. Porém, para resolver esse problemas, seja pelo método analítico, seja pelo algoritmo Simplex, a formulação do modelo deve estar na forma padrão de igualdades (FPI).
\vone
Há portanto três passos básicos a serem realizados para transformar uma PL para a FPI.
\begin{itemize}
    \item[(i)] Trocar min por max, se necessário.
    \item[(ii)] Trocar todas as inequações por igualdades adicionando variáveis extras (sempre não negativas). Estas serão chamadas de \textbf{variáveis de folga}.
    \item[(iii)] Trocar cada \textbf{variável livre} (que não tem restrição se sinal) por duas variáveis não-negativas.
\end{itemize}
\end{ftst}

%==================================

\begin{ftst}{Forma padrão de igualdades}{Programação Linear}
\begin{itemize}
    \item[(i)] Trocar min por max, se necessário.
\end{itemize}
\vone
\centering $\text{min} \ x + y + z$
\begin{figure}
    \centering
    \includegraphics[scale=0.08]{Figuras/seta.png}
\end{figure}

\centering $\text{max} \ -x - y - z$

\end{ftst}


%==================================

\begin{ftst}{Forma padrão de igualdades}{Programação Linear}
\begin{itemize}
    \item[(ii)] Trocar todas as inequações por igualdades adicionando variáveis extras (sempre não negativas). Estas serão chamadas de \textbf{variáveis de folga}. Com exceção das restrições de não-negatividade.
\end{itemize}
\vone
\begin{minipage}{0.45\linewidth}
        \centering $2x + 3y \leq 5$
        \begin{figure}[H]
            \centering
            \includegraphics[scale=0.05]{Figuras/seta.png}
        \end{figure}
        \centering $2x + 3y + w_1 = 5$
        
        \centering $w_1 \geq 0$
\end{minipage}
\hspace{0.05\linewidth}
\begin{minipage}{0.45\linewidth}
        \centering $8x - y + z \geq 10$
        \begin{figure}[H]
            \centering
            \includegraphics[scale=0.05]{Figuras/seta.png}
        \end{figure}
        \centering $8x - y + z - w_2 = 10$
        
        \centering $w_2 \geq 0$
\end{minipage}
\end{ftst}

%==================================

\begin{ftst}{Forma padrão de igualdades}{Programação Linear}
\begin{itemize}
    \item[(iii)] Trocar cada \textbf{variável livre} (que não tem restrição de sinal) por duas variáveis não-negativas.
\end{itemize}
\vone
\begin{minipage}{0.35\linewidth}
    \begin{align*}
        \text{min} \ \ \ & x + y  \\ 
        \text{sujeito a} \ \ \  & x - y \leq 2 \\
        \text{} \ \ \ & x + y \geq -1 \\
        \text{} \ \ \ & x \geq 0
    \end{align*}
\end{minipage}
\hspace{0.02\linewidth}
\begin{minipage}{0.1\linewidth}
    \begin{figure}
        \centering
        \includegraphics[angle=90, scale=0.05]{Figuras/seta.png}
    \end{figure}
\end{minipage}
\hspace{0.05\linewidth}
\begin{minipage}{0.40\linewidth}
    \begin{align*}
        \text{min} \ \ \ & x + y^+ - y^-  \\ 
        \text{sujeito a} \ \ \  & x - (y^+ - y^-) \leq 2 \\
        \text{} \ \ \ & x + (y^+ - y^-) \geq -1 \\
        \text{} \ \ \ & x, y^+, y^- \geq 0
    \end{align*}
\end{minipage}

\end{ftst}

%==================================

\begin{ftst}{Forma padrão de igualdades}{Programação Linear}
O problema FPI de programação linear também pode ser escrito de forma matricial:

\begin{align*}
\text{max} \ \ \ & \textbf{c}^T \textbf{x} \\ 
\text{sujeito a} \ \ \ & \textbf{Ax} = \textbf{b} \\ 
\text{} \ \ \ & \textbf{x} \geq \textbf{0}
\end{align*}
em que:
\footnotesize
\begin{equation*}
\textbf{A} = \begin{bmatrix}
a_{11} & a_{12} & \cdots  & a_{1n}\\ 
a_{21} & a_{22} & \cdots  & a_{2n}\\ 
\vdots & \vdots & \cdots  & \vdots\\ 
a_{m1} & a_{m2} & \cdots  & a_{mn}\\ 
\end{bmatrix}, 
\textbf{x} = \begin{bmatrix}
x_1 \\
x_2 \\
\vdots \\
x_n
\end{bmatrix},
\textbf{b} = \begin{bmatrix}
b_1 \\
b_2 \\
\vdots \\
b_m
\end{bmatrix},
\textbf{c} = \begin{bmatrix}
c_1 & c_2 $ \cdots & c_n
\end{bmatrix},
\textbf{0} = \begin{bmatrix}
0 \\
0 \\
\vdots \\
0
\end{bmatrix}
    
\end{equation*}
\end{ftst}

%==================================

\begin{ftst}{Forma padrão de igualdades}{Programação Linear}
Vamos passar o modelo abaixo para a forma FPI:
\vone
    \begin{align*}
        \text{min} \ \ \ & x + y  \\ 
        \text{sujeito a} \ \ \  & x - y \leq 2 \\
        \text{} \ \ \ & x + y \geq -1 \\
        \text{} \ \ \ & x \geq 0
    \end{align*}

\end{ftst}

%==================================

\begin{ftst}{Forma padrão de igualdades}{Programação Linear}
Agora vamos passar a FPI para a forma matricial:

\begin{align*}
        \text{max} \ \ \ & - x - (y^+ - y^-)  \\ 
        \text{sujeito a} \ \ \  & x - (y^+ - y^-) + w_1 = 2 \\
        \text{} \ \ \ & x + (y^+ - y^-) - w_2 = -1 \\
        \text{} \ \ \ & x, w_1, w_2, y^+, y^- \geq 0
\end{align*}

\end{ftst}

%==================================

\begin{ftst}{Forma padrão de igualdades}{Programação Linear}
Forma matricial:

\begin{equation*}
    \text{max} \ \ \begin{bmatrix}
-1 & -1 & 1 & 0 & 0
\end{bmatrix} \cdot \begin{bmatrix}
x \\ 
y^+\\ 
y^-\\ 
w_1\\ 
w_2
\end{bmatrix} 
\end{equation*}

\begin{equation*}
\text{max} \ \ \begin{bmatrix}
1 & -1 & 1 & 1 & 0 \\
1 & 1 & -1 & 0 & -1 
\end{bmatrix} \cdot \begin{bmatrix}
x \\ 
y^+\\ 
y^-\\ 
w_1\\ 
w_2
\end{bmatrix} = \begin{bmatrix}
2\\ 
-1
\end{bmatrix}
\end{equation*}

\begin{equation*}
    x, y^+, y^-, w_1, w_2 \geq 0
\end{equation*}

\end{ftst}


%==================================

\begin{ftst}{Modelagem: exemplo 4}{Programação Linear}
Transforme o problema linear a seguir para a FPI e para a forma matricial:

\begin{align*}
        \text{min} \ \ z = \ \ \ & 3X_1 + 4X_2 - 5X_3 \\ 
        \text{sujeito a} \ \ \  & 2X_1 + 2X_2 + 4X_3 \geq 320 \\
        \text{} \ \ \ & 3X_1 + 4X_2 + 5X_3 \leq 580 \\
        \text{} \ \ \ & X_1, X_2, X_3 \geq 0
\end{align*}


\end{ftst}

%==================================

\begin{ftst}{Modelagem: exemplo 5}{Programação Linear}
Transforme o problema linear a seguir para a FPI e para a forma matricial:

\begin{align*}
        \text{max} \ \ z = \ \ \ & 5X_1 + 2X_2 - 4X_3 - X_4\\ 
        \text{sujeito a} \ \ \  & X_1 + 2X_2 - X_4 \leq 12 \\
        \text{} \ \ \ & 2X_1 + X_2 + 3X_3 \geq 6 \\
        \text{} \ \ \ & X_1 \text{(\textit{livre})}, X_2, X_3, X_4 \geq 0
\end{align*}


\end{ftst}

%==================================

\begin{ftst}{Modelagem: exemplo 6}{Programação Linear}
Transforme o problema do exemplo 3 para a FPI e para a forma matricial:

\begin{align*}
        \text{min} \ \ z = \ \ \ & 0.14X_1 + 0.12X_2 + 0.20X_3 + 0.75X_4 + 0.15X_5\\ 
        \text{sujeito a} \ \ \  & 23X_1 + 171X_2 + 65X_3 + 112X_4 + 188X_5 \geq 2000 \\
                  \text{} \ \ \ & 0.1X_1 + 0.2X_2 + 9.3X_4 + 16X_5 \geq 50 \\
                  \text{} \ \ \ & 0.6X_1 + 3.7X_2 + 2.2X_3 + 7X_4 + 7.7X_5 \geq 100 \\
                  \text{} \ \ \ & 6X_1 + 30X_2 + 13X_3 + 2X_5 \geq 250 \\
                  \text{} \ \ \ & X_1, X_2, X_3, X_4 \geq 0
\end{align*}
\end{ftst}

%==================================

\begin{ftst}{Modelagem: exemplo 7}{Programação Linear}
Transforme o problema do exemplo 2 para a FPI e para a forma matricial:

\begin{align*}
        \text{min} \ \ z = \ \ \ & 0.75X_1 + 0.72X_2 + 0.92X_3 + 0.90X_4 \\ 
        \text{sujeito a} \ \ \  & X_1 + Y_1 - 500 = Y_2 \\
                  \text{} \ \ \ & X_2 + Y_2 - 8000 = Y_3 \\
                  \text{} \ \ \ & X_3 + Y_3 - 9000 = Y_4 \\
                  \text{} \ \ \ & X_4 + Y_4 = 6000 \\
                  \text{} \ \ \ & Y_1, Y_2, Y_3, Y_4 \leq 4000 \\
                  \text{} \ \ \ & X_1, X_2, X_3, X_4, Y_1, Y_2, Y_3, Y_4 \geq 0
\end{align*}
\end{ftst}

\end{document}