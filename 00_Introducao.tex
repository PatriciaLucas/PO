\documentclass[t]{beamer}

% Load general definitions
\input{preamble.tex}

% Specific definitions
\title[]{Pesquisa Operacional}
\subtitle[]{Introdução à Pesquisa Operacional}
\author[]{Patrícia Lucas\\{\footnotesize }}
\institute{Bacharelado em Sistemas de Informação \\ IFNMG  - Campus Salinas}
\date{\scriptsize Salinas\\Dezembro 2021}

\begin{document}

% cover page
\setbeamertemplate{footline}{}
\begin{frame}

\begin{center}
\includegraphics[width=.15\textwidth]{}
\end{center}
  \titlepage
  \begin{tikzpicture}[remember picture,overlay]
  \node[anchor=south east,xshift=-5pt,yshift=5pt] at (current page.south east) {\tiny Versão 1.2021};
  \node[anchor=south west,yshift=0pt] at (current page.south west) {\includegraphics[width=.25\textwidth]{Logos/salinas_horizontal_jpg.jpg}};
  \end{tikzpicture}  
\end{frame}

% Main slides
%==================================


\begin{ftst}{Introdução à Pesquisa Operacional}{Visão geral}
\small
\begin{itemize}
    \item A Pesquisa Operacional, ou simplesmente PO, surgiu na Inglaterra durante a Segunda Guerra Mundial (1939-1945) para a solução de problemas de natureza logística, tática e de estratégia militar, marcando a primeira atividade formal desse campo de estudo. 
    \vone
    \item Os resultados positivos alcançados fizeram com que a Pesquisa Operacional fosse disseminada nos Estados Unidos e, em 1947, a equipe liderada por George B. Dantzig deu origem ao método Simplex para resolução de problemas de programação linear.
    \vone
    \item Desde então, esse conhecimento vem sendo aplicado, com sucesso, para a otimização de recursos em diversos segmentos industriais e comerciais de várias áreas de negócio (estratégia, marketing, finanças, microeconomia, operações e logística, recursos humanos, entre outras).
\end{itemize}

\end{ftst}

%==================================

\begin{ftst}{Introdução à Pesquisa Operacional}{Visão geral}
\begin{itemize}
    \item O avanço da Pesquisa Operacional tornou-se possível graças ao aumento da velocidade de processamento e à quantidade de memória de computadores nos últimos anos, tornando possível a solução de problemas complexos. 
    \vone
    \item Um profissional de PO deve ser capaz de identificar a técnica mais apropriada para a solução de determinado tipo de problema, os objetivos para a melhoria, as limitações físicas e computacionais do sistema, sendo o elemento humano fundamental nesse processo.
\end{itemize}

\end{ftst}

%==================================

\begin{ftst}{Introdução à Pesquisa Operacional}{Visão geral}
\begin{itemize}
    \item Em termos gerais, podemos dizer que a Pesquisa Operacional consiste na utilização de um método científico (modelos matemáticos, estatísticos e algoritmos computacionais) para a tomada de decisões.
    \vone
    \item Dessa forma, a PO atua cada vez mais em um ramo multidisciplinar, envolvendo áreas de engenharia de produção, matemática aplicada, ciência da computação e gestão de negócios.
\end{itemize}
\end{ftst}

%==================================

\begin{ftst}{Introdução à Pesquisa Operacional}{O processo de tomada de decisão}
\begin{itemize}
    \item \textbf{Decisão:} o processo de análise entre várias alternativas disponíveis do curso de ação que a pessoa deverá seguir.
    \item \textbf{Tomada de decisão:} o processo pelo qual são escolhidas algumas ou apenas uma entre muitas alternativas para as ações a serem realizadas.
    \item \textbf{Exemplo:} a escolha de uma alternativa de localização dentre várias disponíveis.
\end{itemize}

\end{ftst}

%==================================

\begin{ftst}{Introdução à Pesquisa Operacional}{O processo de tomada de decisão}
\begin{itemize}
    \item A tomada de decisões é um processo complexo e envolve diversos fatores internos e externos ligados à organização:
    \begin{figure}
        \centering
        \includegraphics[scale=0.4]{Figuras/tomada_decisao.jpg}
    \end{figure}
\end{itemize}

\end{ftst}

%==================================

\begin{ftst}{Introdução à Pesquisa Operacional}{O processo de tomada de decisão}
\small
\begin{itemize}
    \item Ainda existem profissionais e executivos de mercado que insistem em tomar suas decisões sem qualquer embasamento proveniente de um tratamento de dados e sem a consideração de incertezas, riscos e complexidades inerentes ao processo.
    \item O correto tratamento e a adequada análise dos dados podem propiciar, ao tomador de decisão, informações mais precisas e confiáveis que, quando confrontadas com outras informações ou submetidas a provas existentes e a restrições impostas pelo sistema, oferecem o diferencial do conhecimento.
    \begin{figure}
        \centering
        \includegraphics[scale=0.3]{Figuras/conhecimento.jpg}
    \end{figure}
    \end{itemize}

\end{ftst}

%==================================

\begin{ftst}{Introdução à Pesquisa Operacional}{Modelagem para Tomada de Decisão}
\small
\begin{itemize}
    \item Um modelo é a representação simplificada de um sistema real, podendo ser um projeto já existente ou um projeto futuro.
    \vone
    \item No primeiro caso, pretende-se reproduzir o funcionamento do sistema real existente, de forma a aumentar a produtividade, enquanto no segundo caso o objetivo é definir a estrutura ideal do futuro sistema. 
    \vone
    \item O comportamento de um sistema real é influenciado por diversas variáveis envolvidas no processo de tomada de decisão. Devido à grande complexidade desse sistema, torna-se necessária a sua simplificação, a partir de um modelo, de forma que as principais variáveis envolvidas no sistema ou projeto que se pretende entender ou controlar sejam consideradas na sua construção.
\end{itemize}

\end{ftst}

%==================================

\begin{ftst}{Introdução à Pesquisa Operacional}{Modelagem para Tomada de Decisão}

\begin{itemize}
    \item Um modelo é composto por três elementos principais: 
    \vone
    \begin{itemize}
        \item[a.] Variáveis de Decisão e Parâmetros;
        \vone
        \item[b.] Função Objetivo;
        \vone
        \item[c.] Restrições.
    \end{itemize}
    
\end{itemize}

\end{ftst}

%==================================

\begin{ftst}{Introdução à Pesquisa Operacional}{Modelagem para Tomada de Decisão}
\small
\begin{itemize}
    \item[\textbf{a.}] \textbf{Variáveis de Decisão e Parâmetros: }são as incógnitas, ou valores desconhecidos, que serão determinados pela solução do modelo. Representam efetivamente a decisão que deve ser tomada no problema modelado.
    
    \item  \textbf{Escalas de mensuração:}
    \begin{itemize}
        \footnotesize
        \item As variáveis \textbf{contínuas} podem assumir quaisquer valores em um intervalo de números reais (conjunto infinito ou não enumerável de valores). Exemplo: quantidade ótima a ser produzida (em litros) de cada tipo de refrigerante em uma empresa de bebidas.
        \item As variáveis \textbf{discretas} podem assumir valores dentro de um conjunto finito ou uma quantidade enumerável de valores, sendo aquelas provenientes de determinada contagem. Exemplo:  número ideal de funcionários por turno de trabalho.
        \item As variáveis \textbf{binárias} podem assumir dois possíveis valores: 1 (quando a característica de interesse está presente na variável) ou 0 (caso contrário). Exemplo: fabricar ou não determinado produto.
    \end{itemize}
\end{itemize}

\end{ftst}

%==================================

\begin{ftst}{Introdução à Pesquisa Operacional}{Modelagem para Tomada de Decisão}
\small
\begin{itemize}
    \item[\textbf{b.}] \textbf{Parâmetros:} são os valores fixos previamente conhecidos do problema.
    \vone
    \item Exemplo: custo por funcionário contratado.
\end{itemize}

\end{ftst}

%==================================

\begin{ftst}{Introdução à Pesquisa Operacional}{Modelagem para Tomada de Decisão}
\small
\begin{itemize}
    \item[\textbf{c.}] \textbf{Função Objetivo:} é uma função matemática que determina o valor-alvo que se pretende alcançar ou a qualidade da solução, em função das variáveis de decisão e dos parâmetros, podendo ser uma função de \textit{maximização} (lucro, receita, utilidade, nível de serviço, riqueza, expectativa de vida, entre outros atributos) ou de \textit{minimização} (custo, risco, erro, entre outros).
    \vone
    \item Exemplo: minimização do custo total de produção de diversos tipos de chocolates.
\end{itemize}

\end{ftst}

%==================================

\begin{ftst}{Introdução à Pesquisa Operacional}{Modelagem para Tomada de Decisão}

\begin{itemize}
    \item[\textbf{d.}] \textbf{Função Objetivo:} é uma função matemática que determina o valor-alvo que se pretende alcançar ou a qualidade da solução, em função das variáveis de decisão e dos parâmetros, podendo ser uma função de \textit{maximização} ou de \textit{minimização}.
    \vone
    \item Exemplo: minimização do custo total de produção de diversos tipos de chocolates, maximização do lucro líquido na fabricação de diversos tipos de refrigerantes.
\end{itemize}

\end{ftst}

%==================================

\begin{ftst}{Introdução à Pesquisa Operacional}{Modelagem para Tomada de Decisão}

\begin{itemize}
    \item[\textbf{c.}] \textbf{Restrições:} são um conjunto de equações (expressões matemáticas de igualdade) e inequações (expressões matemáticas de desigualdade) que as variáveis de decisão do modelo devem satisfazer. 
    \vone
    \item As restrições são adicionadas ao modelo de forma a considerar as limitações físicas do sistema, e afetam diretamente os valores das variáveis de decisão. 
    \vone
    \item Exemplo: capacidade máxima de produção, demanda mínima aceitável de um produto.
\end{itemize}

\end{ftst}

%==================================

\begin{ftst}{Introdução à Pesquisa Operacional}{Modelagem para Tomada de Decisão}

\textbf{Exemplo:} Ache o máximo da função $f(x_1,x_2) = x_1 + x_2$, supondo que $x_1$ e $x_2$ satisfaçam:
\begin{equation*}
    x_1 \ge 0 \ ; \ x_2 \ge 0 \ ; \ 2x_1 + x_2 \le 4 \ ; \ x_1 + 2x_2 \le 3
\end{equation*}
\vone
Quais são as variáveis de decisão? 
\vone
Qual é a função objetivo? 
\vone
Esse problema é restrito ou irrestrito? 
\vone
Se restrito, quais são as restrições?

\end{ftst}

%==================================

\begin{ftst}{Introdução à Pesquisa Operacional}{Classificação das ferramentas da Pesquisa Operacional}
\begin{itemize}
    \item Os modelos determinísticos são aqueles em que todas as variáveis envolvida em sua formulação são constantes e conhecidas. Os modelos determinísticos são frequentemente resolvidos por métodos analíticos (sistema de equações) que geram a solução ótima.
\end{itemize}

\begin{figure}
    \centering
    \includegraphics[scale=0.5]{Figuras/deterministicos.jpg}
\end{figure}
\end{ftst}

%==================================

\begin{ftst}{Introdução à Pesquisa Operacional}{Classificação das ferramentas da Pesquisa Operacional}
\begin{itemize}
    \item Os modelos estocásticos utilizam uma ou mais variáveis aleatórias em que pelo menos uma de suas características operacionais é definida por meio de funções de probabilidade. Dessa forma, os modelos estocásticos geram mais de uma solução e buscam analisar os diferentes cenários, não tendo a garantia da solução ótima.
\end{itemize}

\begin{figure}
    \centering
    \includegraphics[scale=0.5]{Figuras/estocásticos.jpg}
\end{figure}
\end{ftst}

%==================================

\begin{ftst}{Introdução à Pesquisa Operacional}{Classificação das ferramentas da Pesquisa Operacional}

\begin{figure}
    \centering
    \includegraphics[scale=0.5]{Figuras/outras_tecnicas.jpg}
\end{figure}
\end{ftst}

%==================================

\end{document}

